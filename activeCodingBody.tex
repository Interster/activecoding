Active Coding handboek

Inleiding

Daar is baie tale soos python C Pascal C# Fortran Lisp Visual Basic ens.  Python is maklik om te leer en dek al die basiese aspekte van programmering. Dit is ook baie goed met data wetenskap en Masjienleer.

Programmering is net 'n kommunikasie middel soos enige taal.  Kommunikeer net in hierdie geval met masjiene.

Programmering het baie meer toeganklik geword a.g.v. goedkoop hardeware en die internet.  Github het 40miljoen programmeerders. Dit is maar 0.6% van die w�reldpopulasie. Dus is daar baie ruimte.

Programmering behels vanaf die uitwerk van 'n algoritme met pseudokode tot die vertaling van pseudokode in 'n taal. Tot die bestuur van die weergawes van die kode met 'n sagteware argiveringstelsel soos git.

Hierdie handleiding fokus op python.

In die komende jare gaan die vaardigheid om vinnig te kan tik en om te kan programmeer wees soos om geletterd te wees in die verle� dae.  Met masjienleer en AI en die internet in die 4de industri�le rewolusie is dit die sleutelvaardighede vir die nabye toekoms.

Oopbronsagteware en Linux is ook baie belangrik omdat dit rekenaarwetenskap goedkoop maak.  Dit maak dit toeganklik want dit loop op byna enige hardeware.  Jy kan ook maklik leer deur voorbeeldkode.

Moderne rekenaars is oral en oral beskikbaar.  Soos fone wat meer prosesseringsvermo� het as die maanlandings rekenaar.


1.  Pseudokode

Gee voorbeeld deur 'n while lus te doen met mense. 'n teller tel op tot 3 waarna die persoon na regs gestuur word indien teller bo 3 is.

2.  Veranderlikes

Stoor inligting in veranderlikes. Soos 'n nommer in bokse te plaas.

Wys vergelyking van a + b
Waar a = 2 en b = 3.  Antwoord is 5

Wys operateurs


3.  Die if stelling

Verduidelik die stelling

Boolse logika vinnig verduidelik

4.  Die skikking datastruktuur

Klomp boksies in 'n ry gevul met nommers


5.  Die for stelling

Gebruik die for stelling om 'n skikking te kwadreer


6.  Die while stelling

Verduidelik dit met die borrelsorteer algoritme


7.  Funksies

Funksies breek moeilike take op in 'n klomp klein takies.

Funksie definisie in python

Klein funksie om line�re funksie waarde te bereken. y = mx + c


7.10  Rekursie


8.  Prosedurele programmering

Programme wat bestaan uit klomp funksies met een ingangspunt in die program.


9.  Objek gebaseerde programmering

Model van objek met funksies wat geroep word as iets met objek gebeur.

Definieer sirkel as objek.  Bereken area en die omtrek as die metodes van die objek.  Die eienskap is die radius van sirkel.




10.  Netwerke

Die internet en hoe netwerke werk in neutedop


11.  Sagteware argivering met git

Geskiedenis van git. Linus Torvalds skryf Linux en kon nie kommersi�le sagteware kry wat werk nie en toe skep hy git

Algemene gebruik van git

Clone en commit

12.  Dokumentasie

Markdown 

Voorbeeld van dokument met formule

LaTeX

Voorbeeld van dokument met formule
